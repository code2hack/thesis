\chapter{相关工作介绍}
\section{分层存储}

所谓分层存储,是指由两种或更多种类型的存储组成的数据存储环境,其特点是各级存储之间的价格,性能,容量和功能这四个主要属性存在差异,从而在存储系统中扮演不同角色。近年来,随着存储容量的逐年提升和存储技术的迅速发展,这种层次化特点越来越鲜明:底端存储性能低,但具有超高容量和极低成本的优点;上层存储则具备非常高的性能水平和强大的数据管理功能,具备自动化数据管理功能的分层存储环境已成为一种必要的体系结构。分层存储系统的设计理念可以概括为:能够实现自动化的数据迁移,快速响应应用层的数据需求,以最低的代价获得最理想的综合性能。

%实际上,将效率与成本相同的存储介质部署在不同层级进行数据迁移复制在性能及成本上并不是有效的数据存储方式。因此,分层存储结构使用有差别的存储介质,以期在相同成本下,既满足性能的需要又满足容量的需要。这种存储介质上的差别主要是在存取速度上及容量上。存取速度快的介质通常都是存储单位成本(每单位存储容量成本,如1元/GB)高,而且容量相对来讲比较低。相应的,存取速度慢的介质通常是为了满足容量与成本方面的要求,既在相同的成本下可以得到更大的容量。所以,从这方面来说,分层存储其实是一种在高速小容量层级的介质层与低速大容量层级的介质层之间进行一种自动或者手动数据迁移、复制、管理等操作的一种存储技术及方案。

%此处应有举例说明{\color{orange}主要的存储供应商和许多新的存储厂商已经宣布计划或提供各种分层存储解决方案。 实际上,很少有供应商提供完整的分层存储产品组合,包括高性能SSD(固态磁盘),RAID阵列和归档磁带库。 实际上,许多供应商的分层产品都是“仅磁盘”策略,因为它们仅包括磁盘产品的RPM速度和价格范围的变化。 尽管这是大多数存储供应商都流行的分层存储方法,但是由于它迫使存档,活动量较小的数据驻留在不断旋转的磁盘上,因此无法有效地服务于第3层数据。 未使用的数据不应消耗能量。 请注意,企业不一定需要使用每个可用层,但是存储池越大,分层存储的好处就越大。}

\subsection{分层存储模型}
甲骨文公司关于分层存储研究报告\cite{Tiered_Storage_Takes_Center_Stage}指出,当前主流的分层存储模型可分为4个层级:

第0级:高性能存储。该层级是最接近计算结点的存储设备,存储的内容包括HPC应用,高性能数据库,数据库加速,索引、日志、卷文件、元数据存储等。该层级的存储介质通常为SSD(DRAM或闪存)。SSD提供极高的I/O性能,同时单位存储的造价也最高。

第1级:主要存储。主要存储层的存储内容为任务相关的关键数据,通常采用光纤通道技术(FC,Fibre Channel)将磁盘阵列连接组成区域存储网络(SAN,Storage Area Network)。该层级存储提供高性能、低延迟、高可用性和快速数据恢复等特性,可以快速稳定地为各类任务提供存储服务。

第2级:次级存储。次级存储的主要任务是存储相对重要的数据,例如常规的数据库、数据备份等等。该层次的数据I/O性能要求相对较低,注重易用性、易管理性、可扩展性和相对较低的成本,因此通常采用以太网连接磁盘阵列组成附加存储网络(NAS,Network Attached Storage)。

第3级:长期存储。该层级是存储系统中的最底层,存储对象是长期存档数据,例如社交网络存档数据、安保视频数据以及大规模科学计算中的归档数据等。此类数据总容量极大,访问频率低,具有一次写入多次读取(Write-Once-Read-Many)的特点,几乎没有I/O性能要求,因此磁带存储(Tape Storage)和近年兴起的蓝光光盘是最理想的存储介质。

\begin{table}[htbp]
\centering
\begin{minipage}[t]{0.9\linewidth}
\caption{典型分层存储模型}
\label{tab:TS_model}
\begin{tabularx}{\linewidth}{cZcZcZcZ}
\toprule[1.5pt]
{\hei 存储层级} & {\hei 第0级} & {\hei 第1级} & {\hei 第2级} & {\hei 第3级} \\
\midrule[1pt]
容量占比 & 1-3\% & 12-20\% & 20-25\% & 43-60\% \\
\hline
主要存储技术 & SSD & FC-SAN & NAS & 磁带库,蓝光存储 \\
\hline
数据类型 & I/O密集型 & 任务关联型 & 重要、敏感 & 长期归档 \\
\hline
I/O性能要求(IOPs) & >$10^6$ & 200-300 & 100-200 & 极低 \\
\hline
容灾指标RPO(小时) & <4 & <4 & <12 & 一天以上 \\
\hline
容灾指标RTO(小时) & <1-2 & <1-2 & <5 & <24 \\
\hline
功耗 & 低 & 最高 & 高 & 最低 \\
\bottomrule[1.5pt]
\end{tabularx}
\end{minipage}
\end{table}

\subsection{数据分类}
对于数据管理者来说,应当应用实际需求对文件数据分类,以匹配存储层级的方式对数据进行管理。与0-3层存储层级相对应,数据可分为以下4类:

I/O密集型数据。此类数据在所有文件中容量占比大约为3\%,主要包括对响应时间要求严苛的应用与文件,由于对I/O性能的优先级最高,因此存放在第0层。例如高性能操作系统文件、HPC应用、高性能数据库以及索引、日志、目录文件等元数据文件。

任务关联型数据。此类数据是指与正在运行的应用直接关联的数据,容量占比大约12-20\%,存放在第0级或第1级以保证应用的正常运行。这类数据对容灾能力要求较高,数据恢复时间(Recovery Time Objective,RTO)通常在几分钟到1-2小时。另一方面,当切换至不同的应用负载时,应及时将相关数据从第2级迁入到第0-1级。任务关联型数据包括任务相关的数据库、虚拟机、在线交易系统等等。


重要与敏感数据。这类数据不会直接影响应用的正常运行,因此通常存储于第2级,容量占比大约20-25\%。容灾能力要求较宽松,通常恢复时间允许在4小时以内。这类数据包括数据库、网络服务和应用、数据备份、恢复与数据安全系统、云存储等。

归档数据。归档数据容量占比最高,约占总容量的43-60\%,也是容量增长速度最快的一类数据。归档数据的访问频率最低,部分具有一次写入多次读取的特点,对I/O性能和容灾要求最低,通常存储在最底层的第3层。例如长期备份数据、离线媒体类数据、非结构化的文档数据以及大规模科学工程计算归档数据等。

\subsection{分层存储管理}
分层存储管理(Hierarchical Storage Management,HSM)是指对存储系统中所有数据统一管理调度,根据数据的访问频率、性能要求等因素进行实时分类,并在不同层级之间自动迁移的技术。在上述分层存储模型与数据分类方式下,分层存储管理方案的优劣直接决定了存储系统的整体性能与性价比。

在实际的分层存储管理方案设计中,4分类问题可化为多次二分类任务,即,任意相邻的两个层级分别视为热存储和冷存储,数据的迁移策略由冷热分类的结果决定。分类指标可总结如下:

I/O性能要求。这里的性能要求特指访问延迟,或者说每秒访问次数(IOPs),例如文件元数据,数据库索引文件等访问延迟越低越好。

任务关联性。任务关联性是指,特定的应用在生命周期中的某一阶段,部分数据是该应用即将访问或频繁访问的,这些数据属于任务关联型数据;而其他数据在较长时间内不会被访问,任务关联性较低。

各级存储容量比例、功耗要求、容灾能力。这些指标属于硬件条件等外部环境,决定了数据分类的具体阈值。例如,高性能存储介质的容量高,意味着更多的元数据文件或任务关联型数据可以存放在第0级或第1级;功耗要求严格,意味着需要避免错误分类,减少数据迁移;容灾恢复时间(RTO)短,意味着这部分数据需要在较高层级的存储介质中进行备份。

在本课题中,我们主要针对任务关联性指标展开研究,主要研究对象是第0-1层与第2层之间的数据二分类与迁移策略。在实际应用背景下,数据的任务关联性是随着程序运行动态变化的,良好的分层管理方案应能侦测应用的长期访问模式,并准确识别热点数据指导数据迁移。为此,在后续研究中,我们针对POSIX文件系统建立了路径嵌入模型,以量化分析文件之间的静态关联;在此基础上,建立循环神经网络分析特定应用负载的I/O行为,挖掘数据与应用运行时的动态关联规律,并以此为依据进行文件的冷热分类实验。

