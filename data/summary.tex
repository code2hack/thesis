\chapter{总结与展望}
在当前大数据与高性能计算蓬勃发展的背景下,分层存储系统在计算机领域的应用越来越广泛。尤其是伴随着NVMe、SSD等高性能存储介质相关技术的飞速发展,分层存储管理中的热数据识别和数据迁移技术成为分层存储管理优化的关键。国内外对于相关领域的优秀成果不断涌现,尤其是主动数据迁移预取方面,许多学者对文件访问模式的本质进行了重要的探索,如数据挖掘领域的频繁序列挖掘算法,自然语言处理领域的N-gram模型和循环神经网络等模型均在该领域有了立足之地。

目前分层存储管理优化方式主要包括两方面:一是挖掘文件的关联性(Block Correlation),二是动态追踪应用的I/O行为并加以分析,实现I/O行为预测(热数据识别),以指导数据迁移模块进行主动的数据预取和缓存替换。以往的研究具有局限性,例如,对文件的关联性挖掘局限于目录树结构表现出的文件关系,没有对文件或目录命名隐含的语义关系予以分析;I/O行为的分析通常只针对短期内的访问规律,没有在长时间跨度上进行分析。为解决上述问题,探索新的分层管理优化方式,本文做了以下几方面的工作:

1)文件的向量化表示。数据块关联(Block Correlation)在文件系统中广泛存在,且这种数据间的联 系通常比较稳定,在目录结构不发生变化的情况下,一般不受工作负载的运行状 态影响。反过来,工作负载访问文件的部分规律是由文件之间的固有关系决定的。 如果能显示地挖掘出文件之间的联系,可以为存储系统的数据分布策略、迁移策 略等提供帮助。本文从文件之间语义关系入手,建立了词嵌入模型将文件和目录映射为向量,将文件之间的关联分析转化为向量运算。

2)基于循环神经网络的文件分类模型。本文对冷热文件分类任务进行了形式化定义,在文件向量化表示的基础上,建立了GRU循环神经网络对应用的I/O序列进行分析和文件分类。

3)在上述两项工作的基础上,本文在GlusterFS架构的基础上设计了一套分层存储管理框架,该框架对Tiering模块、CTR模块以及GFDB数据库等进行了改进设计,将文件向量作为额外的元数据,并将循环神经网络嵌入到CTR模块中代替传统的LRU缓存算法指导数据迁移。

4)以开源工程编译任务作为工作负载,设计相关实验对上述两项模型进行验证。


基于上述模型设计,本课题的实验部分选用开源工程编译任务作为工作负载,设计了一套从实验数据采集到模型测试的具体实施方案,给出了相应的评估标准,最后就实验结果进行了充分的分析:基于循环神经网络的文件分类模型在面向单一进程编译任务时取得了一定的效果,但面对更复杂的多任务编译负载时性能较差。

本课题涉及的交叉知识较多,在课题研究的过程中深感自己的知识储备远远不够,工程能力也尚有欠缺。本课题还有许多不足和需要做的工作:

1)进一步完善课题内关于自然语言处理部分的理论框架。

2)要深入分析目前模型存在的缺陷和不合理之处,对模型设计作进一步完善,

3)将研究的范围作进一步扩展,如对象文件系统、并行文件系统等。

4)完成GLusterFS上分层管理模块的工程实现。