\chapter{分层存储与热点文件识别}
随着存储以每年50%到80%的速度增长,闲置数量的增长
数据是各种规模公司的关注点。分析师和编辑认为60%至80%
存储的数据无效。数据一旦创建,一两个月后就很少被访问。
因此,对存储容量的不断增长的需求为
分层存储方法。
IT组织需要找到管理存储和数据的方法,以从中获得最大价值
存储投资。从历史上看,存储经理专注于维护他们的
组织的数据并确保其可用性。一种常见的方法是将数据视为
整体实体,增长太快,消耗的容量越来越大。
尽管公司意识到并非所有数据都具有同等价值,但存储经理
通常将所有数据视为相等。因此,系统和行为反映了该观点。所有
相似存储中的数据。一种备份模型。具有相同保护的单层存储
水平比较容易。实施分层存储方法需要大量时间投入
和资源。关键,重要和不重要的数据之间没有界限。
帮助存储管理员认识到这些差异并采取措施以解决这些问题。
分层存储体系结构及其驱动程序,数据分类的核心。
数据分类不是分层存储。数据分类是决策过程
识别数据并确定其对组织的价值。分层存储是硬件,
软件和实现数据分类计划的过程。
除非打算使用分层存储架构,否则数据分类毫无意义。它的
如果不先分类,几乎不可能将正确的数据放置在适当的存储层中
您的数据。一种看待它的方法是分层存储是存储决策的实现
在数据分类期间进行。
