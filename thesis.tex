%%
%% This is file `thesis.tex',
%% generated with the docstrip utility.
%%
%% The original source files were:
%%
%% nudtpaper.dtx  (with options: `thesis')
%% 
%% This is a generated file.
%% 
%% Copyright (C) 2018 by TomHeaven <hanlin_tan@nudt.edu.cn>
%% 
%% This file may be distributed and/or modified under the
%% conditions of the LaTeX Project Public License, either version 1.3a
%% of this license or (at your option) any later version.
%% The latest version of this license is in:
%% 
%% http://www.latex-project.org/lppl.txt
%% 
%% and version 1.3a or later is part of all distributions of LaTeX
%% version 2004/10/01 or later.
%% 
%% To produce the documentation run the original source files ending with `.dtx'
%% through LaTeX.
%% 
%% Thanks LiuBenYuan <liubenyuan@gmail.com> for maintainence.
%% Thanks Xue Ruini <xueruini@gmail.com> for the thuthesis class!
%% Thanks sofoot for the original NUDT paper class!
%% 
%1. 规范硕士导言
% \documentclass[master,ttf]{nudtpaper}
%2. 规范博士导言
% \documentclass[doctor,twoside,ttf]{nudtpaper}
%3. 如果使用是Vista
% \documentclass[master,ttf,vista]{nudtpaper}
%4. 建议使用OTF字体获得较好的页面显示效果
%   OTF字体从网上获得,各个系统名称统一,不用加vista选项
%   如果你下载的是最新的(1201)OTF英文字体,建议修改nudtpaper.cls,使用
%   Times New Roman PS Std
% \documentclass[doctor,twoside,otf]{nudtpaper}
%   另外,新版的论文模板提供了方正字体选项FZ,效果也不错哦
% \documentclass[doctor,twoside,fz]{nudtpaper}
%5. 如果想生成盲评,传递anon即可,仍需修改个人成果部分
% \documentclass[master,otf,anon]{nudtpaper}
%
\documentclass[master,otf]{nudtpaper}

%%----DELETE-----%
%\usepackage{notes}
%\numberwithin{equation}{chapter}
%\numberwithin{figure}{chapter}
%%----DELETE-----%
%\usepackage{pdfpages}

\usepackage{lmodern}
\usepackage{mynudt}
\usepackage{multirow,array}

\classification{TP399}
\serialno{17060062}
\confidentiality{公开}
\UDC{004.8}
\title{基于自然语言处理的热点数据识别\\及应用技术研究}
%\title{基于自然语言模型的文件访问模式分析与文件预取方法研究}
\displaytitle{基于自然语言处理的热点数据识别及应用技术研究}
\author{陈辉}
\zhdate{\zhtoday}
\entitle{Research on Hot File Identification and Application Technology Based on Natural Language Processing}
\enauthor{Hui Chen}
\endate{\entoday}
\subject{计算机科学与技术}
\ensubject{Computer Science and Technology}
\researchfield{高性能计算}
\supervisor{周恩强\quad{}研究员}
\cosupervisor{}  % 协助指导教师,没有就空着
\ensupervisor{Prof. Enqiang Zhou}
\encosupervisor{} % 协助指导教师英文,没有就空着
\papertype{工学}
\enpapertype{Engineering}
% 加入makenomenclature命令可用nomencl制作符号列表。

\begin{document}
	\graphicspath{{fig/}}
	% 制作封面,生成目录,插入摘要,插入符号列表 \\
	% 默认符号列表使用denotation.tex,如果要使用nomencl \\
	% 需要注释掉denotation,并取消下面两个命令的注释。 \\
	% cleardoublepage% \\
	% printnomenclature% \\
\maketitle
\frontmatter
\tableofcontents
\listoftables
\listoffigures

\midmatter
\begin{cabstract}
分层存储是计算机存储领域的一项重要技术,其核心设计是将数据存储在多层级的存储介质中,通过热点文件识别和数据迁移技术来掩盖访问延迟以及增加吞吐率。分层存储管理的本质是准确、实时的文件分类,当存储层次较多时,可以转化为相邻层级之间热点数据和冷数据的二分类任务。文件分类的准确率十分依赖于对应用I/O行为的理解。本文工作主要包括以下两部分:

%1. 文件静态关联分析。我们提出了两条假设。第一条假设是POSIX文件系统的目录结构具有与自然语言相似的性质,即分布式语义假设。基于该假设,我们使用词嵌入模型将POSIX文件系统中的文件名、目录名映射为高维向量,以量化的方式分析它们之间的语义关联。第二条假设是一条完整路径的各级子向量可以线性相加,得到的结果可以近似表示该文件在文件系统空间中的位置。基于该假设我们提出了路径向量的概念。
文件关联分析。数据块关联性广泛存在于存储系统中,以往的研究通常从POSIX文件目录的树形结构入手,无法充分挖掘语义上的关联性。本文使用词嵌入模型将POSIX文件系统中的文件和目录映射为高维向量,将文件或目录之间的语义关联转化为向量空间中的内积计算,并通过降维可视化的方式予以直观呈现。
%2. I/O行为分析。在上述静态分析的基础上,我们提出了第三条假设:应用的I/O行为研究,可以转化为路径向量的序列分析问题。基于该假设,我们建立了循环神经网络来分析路径向量序列,最终用于解决文件的冷热分类任务,并利用GlusterFS中的tiring模块设计了一套智能分层存储管理的初步方案。

基于I/O行为分析的冷热文件分类。目前,该领域的研究通常局限于短期I/O请求序列的分析和预测,难以在较长时间跨度上挖掘应用的I/O行为模式。本文在文件和目录向量化表示的基础上,使用循环神经网络分析I/O请求序列,最终用于解决文件的冷热分类任务,并利GlusterFS文件系统中的tiring模块设计了一套智能分层存储管理的初步方案。

为验证模型的合理性,本文以开源工程代码编译为工作负载设计了文件冷热分类实验。实验结果表明,对于单进程编译任务,在经过参数调优后可以达到较高的热点文件识别率,且冷数据被误判的概率控制在较低水平。

\end{cabstract}
\ckeywords{分层存储;数据迁移;访问模式;词嵌入;循环神经网络}

\begin{eabstract}
Tierd storage is an import technology in the field of computer storage, which stores data in multiple tiers, masking latency and increasing throughput through file classification and data migration techniques. In fact, the main purpose of hierarchical storage management is to correctly classify files in runtime. When dealing with multi-tired storage system, this mission can be splitted into several sub-missions of binary classification. However, the correctness of file classification heavily depends on the understanding of the I/O behavior. Our research focuses on two aspects.

%Analysis on file correlations. In this part, we propose two assumptions. The first is the distributional hypothesis that the directory structure of POSIX file system is similar to natural language. Based on this hypothesis, we utilize word embedding model to map file names and directory names to high-dimension vectors, analyzing the semantic correlations among those names in a quantified manner. The other is that we could sum up those sub-vectors of a full path to approximately represent this path in the space of file system. Thus, we propose a concept named path vector. 

Analysis on file correlations. File correlations are common in storage systems. Recent researches in this area often focus on the tree structured directory of POSIX file systems which were unable to reveal the implicit semantic correlations among files and directories. We utilize word embedding model to map file names and directory names to high-dimension vectors, analyzing the semantic correlations among them by dot products. With the word embedding model trained, we show the semantic correlations with PCA and visualization.

%Sequential analysis on I/O behavior. Based on aforementioned works, we propose the third assumption, that the research on I/O behavior could be reframed to sequential analysis on path vectors. We build an RNN model to achieve this goal and design a hierarchical storage management solution based on tiering translator in GlusterFS.

File classification based on sequential analysis on I/O behavior. Recent works were not capable of long-term analysis. In this paper, based on aforementioned works, we build an RNN model to analyze the I/O behavior and design a hierarchical storage management solution based on tiering translator in GlusterFS.

To verify these models, we designed file calssification experiments with compiling workloads. The results show that our model performs well in single-process compiling, that the accuracy of hotspot indentification(recall rate) is good with fine-tuned hyperparameters, and the false positive rate controlled bellow.
\end{eabstract}
\ekeywords{Tiered Storage, Data Migration, Access Pattern, Word Embedding, RNN}



\mainmatter

\chapter{绪论}
\section{研究背景}
分层存储系统(Tiered Storage System),又称为层级存储管理(Hierarchical Storage Management),是当前各领域存储系统的主流架构。其基本特点是将文件数据存储在若干层级的存储介质中,不同层级的介质(DRAM,SSD,HDD等)具有不同的容量、吞吐率、访问延迟与成本等。不同的应用负载和场景下,存储系统中的数据重要性存在差异,可以粗略地分为“热”数据与“冷”数据:正在访问与频繁访问的“热”数据优先存储于更接近CPU、访问更快的内存、SSD阵列等存储介质中;近期没有访问需求或访问频率低的“冷”数据主要存储在低层级的磁盘阵列或远程存储服务器中。在现实应用中,数据的“热”与“冷”不是静态的,而是随着应用的访问需求动态变化,数据在不同层级之间的迁移管理成为层次存储系统的基本任务之一。
\begin{figure}[htp]
    \centering
    \includegraphics[width=\textwidth]{chp2_multi_tiers}
    \caption{分层存储系统(Tiered Storage System)}
    \label{fig:multi_tiers}
\end{figure}
如图\ref{fig:multi_tiers}所示,分层存储系统按层级从高到低由不同存储介质组成:
高层级存储通常由内存DRAM(Dynamic Random Access Memory)、非易失性存储(Non-volatile Memory)等介质组成,其特点是读写性能优异、存储密度高,是实际应用场景中“热”数据的理想存储载体,同时也具有易失(DRAM)或寿命有限(NVM)等缺点

较低级存储通常由固态硬盘(Solid-State Driver)、磁盘(Hard Disk Drive)阵列组成,且在生产环境下这些存储阵列通常部署在远程存储服务器(Storage Servers),通过SANs(Storage Area Networks)或高速互联网络(InfiniBand)与客户端或计算服务器连接以提供存储服务。低层级存储性能较弱,存储密度较低,但同时也具有造价低廉、容量大、稳定性好的优点,且可以通过冗余磁盘阵列(Redundant Array of Inexpensive Disks)以及数据复制(Replication)等技术进一步提高数据的稳定性和容错能力。因此这类低层级存储是“冷”数据的理想载体。

近几十年来,分层存储在商业性数据存储领域得到了大规模应用。例如NetApp FAS系列存储系统
\cite{NetAppFas},
IBM DS8880
\cite{IBMDS8880}
等。这些支持Flash存储介质的存储系统方案主要可分为两类:第一类存储方案简单地将Flash存储介质作为较内存低一级的缓存
\cite{NetAppFas}
,以起到优化存储系统性能的作用。采用Flash存储介质作为缓存的主要优点是集成容易,不需要显式地考虑数据迁移策略,同时造价比DRAM低得多。第二类方案则是将Flash存储介质作为持续存储加入到层次存储结构中。

在大规模科学与工程计算领域,层次存储结构同样应用广泛。随着闪存技术的飞速发展,传统HPC所应用的三层技术架构(计算结点的共享内存-并行文件系统-归档存储)也随之发生变化。在HPC系统中,并行文件系统(pFS)对HPC性能影响非常大,在许多场景下决定了整个HPC的存储性能。传统HPC架构在应对超大规模HPC集群计算节点同时Checking Point需求时,显得力不从心,那就需要在pFS之上多加一层高速大容量的缓存(Burst Buffer)。

层次存储系统中的不同层级间的数据迁移管理是文件系统性能的关键。可粗略分为被动缓存和主动预取。预取(prefetching),也称为预分页(prepaging)或预读(read-ahead),是操作系统数据读取过程中的重要优化方法
\cite{Reducing_File_System_Latency_using_a_Predictive_Approach}
\cite{Group_based_management_of_distributed_file_caches}
\cite{A_data_mining_algorithm_for_generalized_web_prefetching}
。其目标是预测将来的数据访问,并在请求数据之前将其提前读取到高层级的存储介质中,从而起到掩盖访问延迟的作用。预取是传统缓存技术的一种补充,不同于被动数据迁移,预取技术的关键在于对即将访问的数据内容和生存周期进行主动预测。

热数据预测:主动预取需要预测程序下一阶段可能访问的数据,是对数据访问的空间局部性的扩充。针对“热”数据准确的预测将极大地降低访问延迟,而错误预测将会引发浪费传输带宽,挤占缓存空间等负面影响。

热数据生命周期预测:与缓存机制中的时间局部性类似,主动预取需要针对缓存数据的时效性和生命周期建立有效的评估。非缓存数据的及时预取有助于提高存命中率,而清除短期内不再读取的数据将提高缓存空间的利用率。

%预取的主要流程可总结如下:
%\begin{enumerate}
%\item 针对特定负载,提取访问日志;
%\item 分析访问日志,对该负载的文件访问模式进行抽象表达;
%\item 将负载的访问模式作为依据,引导文件系统进行主动预取。
%\end{enumerate}


\section{国内外研究现状}
%\linkout{overview}{14}
\subsection{缓存优化技术}
缓存优化是计算机存储领域的核心技术之一,近年来国内外涌现了大量相关研究成果。这些成果的主要围绕提升缓存硬件性能、存储架构优化、实现多级缓存等核心技术展开。

Eshel等\cite{Panache}
提出并实现了缓存文件系统“Panache”,该系统使用pNFS以分布式缓存的方式存储GPFS中的缓存数据。
Frings等\cite{Massively_Parallel_Loading}
针对动态链接库加载过程进行数据预取,从而提高并行应用程序的性能。
Rajachandrasekar等\cite{1PB}
提出了一种用户级文件系统,将检查点请求保留在主内存中,并同时写入到持久性存储中。他们的方法包括对远程直接内存访问(RDMA)的支持。

Zhao等\cite{HyCache+}
提出了另一种缓存中间件,采用一种双阶段缓存技术来减少计算结点和I/O结点之间的数据传输量。
Isaila等\cite{Multi_Leve_Data_Staging_for_Blue_Gene}
提出了分别位于客户端和I/O节点之间,以及I/O节点和存储服务器之间的两级预取方案,从而改进了IBM Blue Gene的I/O转发层的数据传输性能。
Prabhakar等\cite{Adaptive_Multi_level_Cache_Allocation_in_Distribute_Storage_Architectures}
通过线性规划对两级缓存系统上的最佳缓存分配进行建模。

Kandemir等\cite{On_Urgency_of_I/O_Operations}
定义了I/O请求紧迫度的概念,该概念由请求可以延迟多长时间而不影响应用程序性能给出。在对I/O请求进行紧迫度分析后,通过优先处理紧急请求来改进缓存机制。
Seelam等\cite{Masking_IO_latency_using_application_level_IO_caching_and_prefetching_on_Blue_Gene_systems}
实现了能够追踪和分析应用I/O访问模式的库,并借此引导预取线程将数据提前读取到本地存储。
Patrick\cite{Cashing_in_on_Hints_for_Better_Prefetching_and_Caching_in_PVFS_and_MPI_IO},
He\cite{KNOWAC},
Tang\cite{Improving_read_performance_with_online_access_pattern_analysis_and_prefetching}
等提出了类似的方法(使用访问模式检测指导预取)。

Suei等\cite{Endurance_Aware_Flash_Cache_Management_for_Storage_Servers}
提出了一种使用SSD作为HDD缓存的存储集群缓存设计。其设计侧重于响应时间和缓存命中率。 

Welch和Noer\cite{Optimizing_a_hybrid_SSD_HDD_HPC_storage_system_based_on_file_size_distributions}
根据并行文件系统中小文件占多数的特点,将小文件存储在SSD中以优化对它们的访问。
He等\cite{Proceedings_of_the_22nd_international_symposium_on_High_performance_parallel_and_distributed_computing}
提出了一个代价模型来辅助数据迁移决策。该模型能够评估文件不同区域的访问成本,并将高成本区域放置在SSD中。

\subsection{主动预取技术}
主动预取技术是另一大类提升分层存储性能的主要技术。与被动缓存不同,主动预取的目标是尽量避免缓存缺失,而是要预测应用下一步要访问的数据,从而提前将数据从低级存储迁入缓存。因此,这方面的研究通常将重点访问预测模型入手。下面将对几项颇具代表性的研究展开介绍。



\subsubsection*{数据关联挖掘}
\begin{figure}[htp]
\centering
\includegraphics[width=\textwidth]{block_correlations}
\caption{数据块关联性的例子}
\label{fig:block_correlations}
\end{figure}
数据块关联性(Block Correlations)广泛存在于存储系统中。如图\ref{fig:block_correlations}所示,两个或以上的数据块存在语义关系时,它们便是相互关联的。如图中的目录块“/dir”与inode块“/dir/foo.txt”直接关联,后者与其对应的文件块直接关联。在这种情况下,目录块“/dir”与“/dir/foo.txt”的文件块间接关联。

需要指出的是,块关联性是静态存在的,与工作负载的运行过程无关,在文件系统结构不发生变化的情况下,这种静态关联能够长时间保持。另一方面,这种静态的关联通常会在工作负载动态访问数据的过程中得到体现,即,多数应用在访问数据块时会遵循块关联关系依次访问。因此,挖掘数据块关联对分析应用访问模式,数据缓存策略以及数据分布策略等技术能够提供极大帮助。

Zhenmin Li等人\cite{c_miner}于2004年提出了用于块数据关联性挖掘的C-Miner算法,该算法衍生于数据挖掘领域中的频繁序列挖掘算法(Frequent Sequence Mining)。
频繁序列挖掘是一种在序列数据库中发现频繁子序列的关联分析方法。当子序列出现在序列数据库中至少指定数目的序列(min support)中时,被认为是频繁的。例如,数据块访问序列数据库具有五个序列:
\begin{equation*}
    D = \{ abced, abcef, abijc, aklc \}
\end{equation*}
若取最小支持数为4,那么该数据库就包括以下频繁序列$\{ ab:4, ac:5, bc:4 ,abc:4 \}$。因此可以说,以上频繁序列中的文件具有关联。

C-Minier算法对传统的频繁序列挖掘算法进行改进,降低了计算复杂度,在性能方面针对文件系统的特有结构进行了优化,该算法在文件预取、数据分布调度等研究领域取得了不同程度的应用。

例如Jianwei Liao等
\cite{Prefetching_on_storage_servers_through_mining_access_patterns_on_blocks}
在存储服务器端应用C-Miner算法挖掘数据块关联,同时将客户端的应用信息嵌入到I/O访问请求中共同传递到服务端,两者结合以提升文件预取性能。Zhang等人\cite{composite_file}设计实现了一种聚合文件系统,其核心技术是将经常连续访问,具有较高关联度的小文件聚合组织成一个大文件,并用新的inode作为这个大文件的索引,在访问其中小文件的同时将聚合的其他小文件一并取出,从而大幅提升小文件访问性能。

然而频繁序列挖掘算法存在一定的缺陷,对于较长的访问序列,频繁子序列挖掘的计算代价太高。
\subsubsection*{N-gram模型}
N-gram模型是自然语言处理中一种基于统计方法的语言模型。其基本思想是在处理文本序列时,定义一个大小为$n$的上下文滑动窗口,每个窗口内截取的文本片段就成为gram。在此基础上,对所有gram出现的次数进行统计,并且按照预设的阈值进行过滤,最后形成gram列表。该模型基于这样的假设:第n个词的出现只与前n-1个词相关,而与其他任何词都无关(即n-1阶马尔科夫性质),整句的概率就是各个词出现概率的乘积。

事实上,N-gram模型与上文提到的频繁序列挖掘具有相似之处,gram相当于频繁子序列,形成gram的阈值相当于最小支持数。区别在于,gram的形成必须是严格相邻的n个元素,而频繁子序列没有这一要求。

Subedi等人\cite{stacker}
在Calibrun超算和Titan Cray XK7平台上,基于DataSpace框架\cite{dataspace}
基础上设计实现了一个名为“Stacker”的分层存储管理框架。
\begin{figure}[htp]
\centering
\includegraphics[width=\textwidth]{stacker}
\caption{多层存储管理框架Stacker的架构}
\label{fig:stacker}
\end{figure}

Stacker主要面向的工作负载是XGC1\cite{XGC1},S3D\cite{S3D}等数据量可达100PB级的模拟仿真计算。仿真结果可视化是这类负载的重要组成部分,过大规模的数据无法完全存放于共享内存中,需要SSD阵列及磁盘阵列组成多级存储,同时为了满足可视化应用需求,数据从SSD预取到内存必须及时、准确,因此数据迁移管理是该存储系统中的技术难点。Stacker的设计目的是通过机器学习的方法捕捉应用的数据访问模式,并利用访问模式的特征指导数据迁移模块进行高效的数据预取和下载。与此同时,在如此复杂的应用背景下,仅仅考虑单应用的访问模式是远远不够的,必须考虑到大规模并行计算应用背景下,多个不同任务进程之间表现的不同的I/O行为和干扰。为此,Stacker针对此类科学计算应用负载进行了优化,采用N-gram模型对应用的访问模式进行识别和预测。预测的主要内容包括访问的变量名,以及变量内部的偏移和大小。

以变量名访问模式为例,当存储服务器收到一个新的读数据请求,Stacker模块将会对此前的n条访问历史进行分析,提取其中的1-gram到n-gram序列并更新gram列表中。与此同时,分别进行n-gram,n-1-gram直至2-gram的查找匹配,最终将列表中出现频率最高的匹配结果作为该时刻之后变量名访问的预测。

N-gram模型的劣势与C-Miner算法类似,随着分析的序列长度增加,匹配的时间复杂度和gram列表存储的空间复杂度会迅速增加,Stacker的设计中gram的长度通常不超过20。因此比较适用于大文件的对象存储,小文件访问性能未能证明其有效性。




\section{本文主要工作}
本文的主要研究目标是分层存储系统中的热点数据识别,通过一系列理论研究和实践,基于文件系统与自然语言处理隐含的相似性建立了词嵌入模型与循环神经网络模型,将热点数据识别问题转化为冷热数据二分类问题展开研究。

目前分层存储管理优化方式主要包括两方面:一是挖掘文件的关联性(Block Correlation),二是动态追踪应用的I/O行为并加以分析,实现I/O行为预测(热数据识别),以指导数据迁移模块进行主动的数据预取和缓存替换。以往的研究具有局限性,例如,对文件的关联性挖掘局限于目录树结构表现出的文件关系,没有对文件或目录命名隐含的语义关系予以分析;I/O行为的分析通常只针对短期内的访问规律,没有在长时间跨度上进行分析。为解决上述问题,探索新的分层管理优化方式,本文做了以下几方面的工作:

1)文件的向量化表示。数据块关联(Block Correlation)在文件系统中广泛存在,且这种数据间的联 系通常比较稳定,在目录结构不发生变化的情况下,一般不受工作负载的运行状 态影响。反过来,工作负载访问文件的部分规律是由文件之间的固有关系决定的。 如果能显示地挖掘出文件之间的联系,可以为存储系统的数据分布策略、迁移策 略等提供帮助。本文从文件之间语义关系入手,建立了词嵌入模型将文件和目录映射为向量,将文件之间的关联分析转化为向量运算。

2)基于循环神经网络的文件分类模型。本文对冷热文件分类任务进行了形式化定义,在文件向量化表示的基础上,建立了GRU循环神经网络对应用的I/O序列进行分析和文件分类。

3)在上述两项工作的基础上,本文在GlusterFS架构的基础上设计了一套分层存储管理框架,该框架对Tiering模块、CTR模块以及GFDB数据库等进行了改进设计,将文件向量作为额外的元数据,并将循环神经网络嵌入到CTR模块中代替传统的LRU缓存算法指导数据迁移。





4)以开源工程编译任务作为工作负载,设计相关实验对上述两项模型进行验证。


\section{论文结构}
本文共包含六个章节,其结构组织编排如下:

第一章绪论,对当前分层存储系统在计算机领域的广泛应用进行了阐述,尤其是伴随着NVMe、SSD等高性能存储介质相关技术的飞速发展,分层存储管理中的热数据识别和数据迁移技术成为分层存储管理优化的关键。本章对相关领域发展概况进行了介绍,重点描述和分析了几项主动预取技术,作为本课题研究的引子。

第二章为相关背景工作介绍。本章在第一节陈述了分层存储模型的定义,并对当前主流的4层存储模型进行了阐述。与4层存储模型相对应的是数据的分类方式,主要包括I/O密集型、任务关联型、重要与敏感数据、归档数据等。在此存储模型下,数据分类的指标可归结为I/O性能要求、任务关联性以及外部环境要求等三方面。本文的研究针对任务关联性指标展开,目的是为了捕捉文件之间的静态关联和较长时间跨度的任务关联性。第二节对自然语言处理领域的两个重要内容进行了介绍,一是词嵌入技术的发展概况和基本原理,尤其是近年来广泛应用的Word2Vec模型。随后,介绍了循环神经网络的基本概况,阐明其在自然语言处理和序列处理中的广泛应用。

第四章是本课题的理论模型部分,内容包括基于词嵌入模型的文件关联分析和基于循环神经网络的文件分类模型。在文件关联分析部分,本文实现了文件的向量化表示,将文件之间的关联分析转化为向量运算。在基于循环神经网络的文件分类模型中,本文对冷热文件分类任务进行了形式化定义,在文件向量化表示的基础上,建立了GRU循环神经网络对应用的I/O序列进行分析和文件分类。


第五章以第四章内容为理论基础,在GlusterFS架构的基础上设计了一套分层存储管理框架,该框架对Tiering模块、CTR模块以及GFDB数据库等进行了改进设计,将文件向量作为额外的元数据,并将循环神经网络嵌入到CTR模块中代替传统的LRU缓存算法指导数据迁移。

第六章为实验部分。在实验涉及的工作负载选取上,充分考虑了开源工程编译任务的特点,将其列为模型验证初期实验的首选,设计了实验数据采集方案和具体的实验实施方案,给出了相应的评估标准,最后就实验结果进行了分析。
% a)	不同存储层次存储介质,其性能和功能特点差异描述(数据、曲线),突出分层的意义。
% b)	缓存技术需要一个总结,阐明其问题缺陷,直接用于热点数据识别的劣势;
% 	主动预取,需要丰富若干文献;
% c)	本章篇幅少,广度和深度不够。



\chapter{相关工作介绍}
\section{分层存储}

所谓分层存储,是指由两种或更多种类型的存储组成的数据存储环境,其特点是各级存储之间的价格,性能,容量和功能这四个主要属性存在差异,从而在存储系统中扮演不同角色。近年来,随着存储容量的逐年提升和存储技术的迅速发展,这种层次化特点越来越鲜明:底端存储性能低,但具有超高容量和极低成本的优点;上层存储则具备非常高的性能水平和强大的数据管理功能,具备自动化数据管理功能的分层存储环境已成为一种必要的体系结构。分层存储系统的设计理念可以概括为:能够实现自动化的数据迁移,快速响应应用层的数据需求,以最低的代价获得最理想的综合性能。

%实际上,将效率与成本相同的存储介质部署在不同层级进行数据迁移复制在性能及成本上并不是有效的数据存储方式。因此,分层存储结构使用有差别的存储介质,以期在相同成本下,既满足性能的需要又满足容量的需要。这种存储介质上的差别主要是在存取速度上及容量上。存取速度快的介质通常都是存储单位成本(每单位存储容量成本,如1元/GB)高,而且容量相对来讲比较低。相应的,存取速度慢的介质通常是为了满足容量与成本方面的要求,既在相同的成本下可以得到更大的容量。所以,从这方面来说,分层存储其实是一种在高速小容量层级的介质层与低速大容量层级的介质层之间进行一种自动或者手动数据迁移、复制、管理等操作的一种存储技术及方案。

%此处应有举例说明{\color{orange}主要的存储供应商和许多新的存储厂商已经宣布计划或提供各种分层存储解决方案。 实际上,很少有供应商提供完整的分层存储产品组合,包括高性能SSD(固态磁盘),RAID阵列和归档磁带库。 实际上,许多供应商的分层产品都是“仅磁盘”策略,因为它们仅包括磁盘产品的RPM速度和价格范围的变化。 尽管这是大多数存储供应商都流行的分层存储方法,但是由于它迫使存档,活动量较小的数据驻留在不断旋转的磁盘上,因此无法有效地服务于第3层数据。 未使用的数据不应消耗能量。 请注意,企业不一定需要使用每个可用层,但是存储池越大,分层存储的好处就越大。}

\subsection{分层存储模型}
甲骨文公司关于分层存储研究报告\cite{Tiered_Storage_Takes_Center_Stage}指出,当前主流的分层存储模型可分为4个层级:

第0级:高性能存储。该层级是最接近计算结点的存储设备,存储的内容包括HPC应用,高性能数据库,数据库加速,索引、日志、卷文件、元数据存储等。该层级的存储介质通常为SSD(DRAM或闪存)。SSD提供极高的I/O性能,同时单位存储的造价也最高。

第1级:主要存储。主要存储层的存储内容为任务相关的关键数据,通常采用光纤通道技术(FC,Fibre Channel)将磁盘阵列连接组成区域存储网络(SAN,Storage Area Network)。该层级存储提供高性能、低延迟、高可用性和快速数据恢复等特性,可以快速稳定地为各类任务提供存储服务。

第2级:次级存储。次级存储的主要任务是存储相对重要的数据,例如常规的数据库、数据备份等等。该层次的数据I/O性能要求相对较低,注重易用性、易管理性、可扩展性和相对较低的成本,因此通常采用以太网连接磁盘阵列组成附加存储网络(NAS,Network Attached Storage)。

第3级:长期存储。该层级是存储系统中的最底层,存储对象是长期存档数据,例如社交网络存档数据、安保视频数据以及大规模科学计算中的归档数据等。此类数据总容量极大,访问频率低,具有一次写入多次读取(Write-Once-Read-Many)的特点,几乎没有I/O性能要求,因此磁带存储(Tape Storage)和近年兴起的蓝光光盘是最理想的存储介质。

\begin{table}[htbp]
\centering
\begin{minipage}[t]{0.9\linewidth}
\caption{典型分层存储模型}
\label{tab:TS_model}
\begin{tabularx}{\linewidth}{cZcZcZcZ}
\toprule[1.5pt]
{\hei 存储层级} & {\hei 第0级} & {\hei 第1级} & {\hei 第2级} & {\hei 第3级} \\
\midrule[1pt]
容量占比 & 1-3\% & 12-20\% & 20-25\% & 43-60\% \\
\hline
主要存储技术 & SSD & FC-SAN & NAS & 磁带库,蓝光存储 \\
\hline
数据类型 & I/O密集型 & 任务关联型 & 重要、敏感 & 长期归档 \\
\hline
I/O性能要求(IOPs) & >$10^6$ & 200-300 & 100-200 & 极低 \\
\hline
容灾指标RPO(小时) & <4 & <4 & <12 & 一天以上 \\
\hline
容灾指标RTO(小时) & <1-2 & <1-2 & <5 & <24 \\
\hline
功耗 & 低 & 最高 & 高 & 最低 \\
\bottomrule[1.5pt]
\end{tabularx}
\end{minipage}
\end{table}

\subsection{数据分类}
对于数据管理者来说,应当应用实际需求对文件数据分类,以匹配存储层级的方式对数据进行管理。与0-3层存储层级相对应,数据可分为以下4类:

I/O密集型数据。此类数据在所有文件中容量占比大约为3\%,主要包括对响应时间要求严苛的应用与文件,由于对I/O性能的优先级最高,因此存放在第0层。例如高性能操作系统文件、HPC应用、高性能数据库以及索引、日志、目录文件等元数据文件。

任务关联型数据。此类数据是指与正在运行的应用直接关联的数据,容量占比大约12-20\%,存放在第0级或第1级以保证应用的正常运行。这类数据对容灾能力要求较高,数据恢复时间(Recovery Time Objective,RTO)通常在几分钟到1-2小时。另一方面,当切换至不同的应用负载时,应及时将相关数据从第2级迁入到第0-1级。任务关联型数据包括任务相关的数据库、虚拟机、在线交易系统等等。


重要与敏感数据。这类数据不会直接影响应用的正常运行,因此通常存储于第2级,容量占比大约20-25\%。容灾能力要求较宽松,通常恢复时间允许在4小时以内。这类数据包括数据库、网络服务和应用、数据备份、恢复与数据安全系统、云存储等。

归档数据。归档数据容量占比最高,约占总容量的43-60\%,也是容量增长速度最快的一类数据。归档数据的访问频率最低,部分具有一次写入多次读取的特点,对I/O性能和容灾要求最低,通常存储在最底层的第3层。例如长期备份数据、离线媒体类数据、非结构化的文档数据以及大规模科学工程计算归档数据等。

\subsection{分层存储管理}
分层存储管理(Hierarchical Storage Management,HSM)是指对存储系统中所有数据统一管理调度,根据数据的访问频率、性能要求等因素进行实时分类,并在不同层级之间自动迁移的技术。在上述分层存储模型与数据分类方式下,分层存储管理方案的优劣直接决定了存储系统的整体性能与性价比。

在实际的分层存储管理方案设计中,4分类问题可化为多次二分类任务,即,任意相邻的两个层级分别视为热存储和冷存储,数据的迁移策略由冷热分类的结果决定。分类指标可总结如下:

I/O性能要求。这里的性能要求特指访问延迟,或者说每秒访问次数(IOPs),例如文件元数据,数据库索引文件等访问延迟越低越好。

任务关联性。任务关联性是指,特定的应用在生命周期中的某一阶段,部分数据是该应用即将访问或频繁访问的,这些数据属于任务关联型数据;而其他数据在较长时间内不会被访问,任务关联性较低。

各级存储容量比例、功耗要求、容灾能力。这些指标属于硬件条件等外部环境,决定了数据分类的具体阈值。例如,高性能存储介质的容量高,意味着更多的元数据文件或任务关联型数据可以存放在第0级或第1级;功耗要求严格,意味着需要避免错误分类,减少数据迁移;容灾恢复时间(RTO)短,意味着这部分数据需要在较高层级的存储介质中进行备份。

在本课题中,我们主要针对任务关联性指标展开研究,主要研究对象是第0-1层与第2层之间的数据二分类与迁移策略。在实际应用背景下,数据的任务关联性是随着程序运行动态变化的,良好的分层管理方案应能侦测应用的长期访问模式,并准确识别热点数据指导数据迁移。为此,在后续研究中,我们针对POSIX文件系统建立了路径嵌入模型,以量化分析文件之间的静态关联;在此基础上,建立循环神经网络分析特定应用负载的I/O行为,挖掘数据与应用运行时的动态关联规律,并以此为依据进行文件的冷热分类实验。


\chapter{自然语言处理相关技术原理}
自然语言处理(NLP)的定义可以简单概括对人类语言进行自动化、智能化分析以及学会人类表达的一系列计算机技术,是一门包含着计算机科学、时间序列分析以及语言学的交叉学科,这些学科既有区别又相互交叉。

1936年A.M.Turing发明了举世闻名的“图灵机”,使数学中的逻辑符号和真实世界之间建立了联系,为后来计算机的蓬勃发展提供了坚实的理论基础。20世纪50年代,在图灵机的计算模型的基础上,自动机理论被提出,是现代计算机科学发展的基础\cite{自然语言处理的历史与现状}。后来Kleene又在自动机理论模型之基础上提出了正则表达式和有限自动机。1956年,Chomsky提出了上下文无关的语法的理论,同年人工智能被发明后,被迅速应用到自然语言处理领域之中。上下文无关语法的提出使得该领域的研究分为了基于推理规则的符号派和基于概率论的随机派\cite{宋一凡2019自然语言处理的发展历史与现状},在之后很多年里分别高速发展。70年代语音识别算法研制成功,隐马尔科夫模型(Hidden Markov Model,HMM)提出并得到了广泛应用\cite{自然语言处理的历史与现状}。

近年来,随着深度学习的飞速发展,自然语言处理领域也取得了诸多重要突破。RonanCollobert等\cite{Natural_language_processing_(almost)_from_scratch}于2011年的研究提出了一个简单的深度学习框架,在许多NLP经典任务中取得了前所未有的性能,如实体命名识别、语义标注和词性标注等。之后,研究人员提出了大量基于复杂深度学习的算法,用于解决有难度的NLP任务。2013年,Mikolv\cite{skipgram}提出了当前NLP领域最重要的模型之一Skip-gram,该模型以出色的性能表现将单词转化为高维向量,为后续如雨后春笋般涌现的自然语言处理模型奠定了基础。

本章后续内容以词嵌入方法、循环神经网络等主流自然语言处理模型为基础,探讨其在文件系统优化,尤其是数据迁移策略中“冷”、“热”文件分类问题的应用。

\section{词嵌入技术}

%\href{https://www.linkresearcher.com/careers/6c7a15b5-236a-40f3-879f-af2ac06c2557}{NLP综述博客}
%\href{https://blog.csdn.net/mawenqi0729/article/details/80698350}{词嵌入博客}

众所周知,在自然语言处理任务中,第一步工作就是用计算机能够理解的方式表示和描述单词,也就是将其用向量表示。通常有两大类表征方式:离散型表示(one-hot)和分布型表示(distributed representation)。

所谓离散型表示是指,在给定词汇表$V$的条件下,每一个单词被表示为一个维度为$|V|$的向量,该词汇表中任意一个单词,唯一地分配一个维度为1,其余维度均为0。例如单词file在词汇表中第二个出现,则其离散型向量表示为:$v_{file} = [0,1,0,\dots,0]$。这种表示方式相当于为每个单词分配了一个唯一的ID。当词汇表较大时,词嵌入的维度将会非常高,并且无法表达词与词之间的关系。

单词的分布式表示基于语义学中的分布式假设(Distributional Hypothesis)\cite{distributional_hypothesis}:
在相似的上下文中出现的单词通常具有相似的含义。例如单词water和coffee常与drink搭配,因此water与coffee具备一定的的相似性。分布式表示的目的就是将单词转化为稠密的向量(与离散型的one-hot向量相对),将人类自然语言中的单词之间的相似性和逻辑关联转化为向量空间的数学关系来处理(例如使用二范数表达单词的相似度)。见图\ref{fig:t_sne}所示例子,Li等人\cite{visualizing}采用t-SNE方法对60维的词嵌入进行降维处理,结果显示,含义相近的单词在向量空间内“距离”也比较接近。
\begin{figure}[htp]
\centering
\includegraphics[width=\textwidth]{t_sne}
\caption{词向量降维后的可视化}
\label{fig:t_sne}
\end{figure}
%{\color{red}词嵌入降维图}\linkout{nlp_book}{107}

%{\color{red}词嵌入相关文献}\linkout{nlp_book}{128}

为实现这种从自然语言到向量空间的映射,自然语言处理发展历史上出现了许多理论和方法,例如Deerwester于1988年提出的潜在语义索引(Latent Semantic Indexing)方法\cite{LSI},以及该作者后续应用奇异值分解(SVD)对共现矩阵降维而实现的潜在语义分析方法(latent semantic analysis)\cite{LSA},在此后多年里被广泛应用于多种NLP任务,如认知模型\cite{cognitive_model},拼写检查\cite{spell_checking},写作评分\cite{essay_grading}等等。

随着近年来深度学习的发展,基于神经网络的自然语言模型开始流行,Bengio分别在2003年\cite{bengio2003}和2006年\cite{bengio2006}发表的成果表明,神经网络语言模型能在单词预测任务中出色地担任词嵌入转换的角色。

2013年,Mikolov提出了著名的连续词袋(CBOW)和Skip-gram模型\cite{skipgram},可以说这两种词嵌入模型的发明引发了NLP领域的深刻变革,至今为止这两种模型组成的Word2Vec方法仍被广泛使用于学术界和工业界。

%\begin{figure}[htp]
%\centering
%\includegraphics[width=\textwidth]{cbow_skipgram}
%\caption{Word2Vec的两种主要模型:CBOW与Skip-gram}
%\label{fig:cbow_skipgram}
%\end{figure}
%
%Word2Vec模型中,主要有Skip-Gram和CBOW两种模型,从直观上理解,Skip-Gram是给定中心词来预测上下文。而CBOW是给定上下文,来预测中心词。
%



\subsection{Skip-gram模型}
\subsection{子词模型}


%Word2Vec模型实际上分为了两个部分,第一部分为建立模型,第二部分是通过模型获取嵌入词向量。Word2Vec的整个建模过程实际上与自编码器(auto-encoder)的思想很相似,即先基于训练数据构建一个神经网络,当这个模型训练好以后,我们并不会用这个训练好的模型处理新的任务,我们真正需要的是这个模型通过训练数据所学得的参数,例如隐层的权重矩阵——后面我们将会看到这些权重在Word2Vec中实际上就是我们试图去学习的词向量。基于训练数据建模的过程,我们给它一个名字叫“Fake Task”,意味着建模并不是我们最终的目的。假如我们给定一个句子“The dog barked at the mailman”。
%


%\subsection{文件名、路径向量化}


\section{循环神经网络}
%简述RNN发展情况
\subsection{循环神经网络基本原理}

循环神经网络的主要用途是处理和预测序列数据。在全连接神经网络或卷积神经网络模型中,网络的结构都是从输入层到隐藏层再到输出层,层与层之间是全连接或者部分连接的,但每层之间的节点是无法连接的。而循环神经网络的隐藏层之间的节点是有连接的,隐藏层的输入不仅包括输入层的输出,还包括上一时刻隐藏层的输出。

\begin{figure}[htp]
\centering
\includegraphics[width=\textwidth]{rnn_1}
\caption{RNN展开后的结构}
\label{fig:rnn_1}
\end{figure}
对于循环神经网络,一个非常重要的概念就是时刻。循环神经网络中每个时刻的输入都会对应一个输出。从图\ref{fig:rnn_1}中可以看到,循环神经网络的主体结构A的输入除了来自输入层$X_t$,还有一个循环的边来提供当前时刻的状态。在每一个时刻,循环神经网络的模块A会读取t时刻的输入$X_t$,并输出一个值$H_t$。同时A的状态会从当前步传递到下一步。因此,循环神经网络理论上可以被看作是同一神经网络结构被无限复制的结果。但出于优化的考虑,目前循环神经网络无法做到真正的无限循环,所以,现实中一般会将循环体展开。

\begin{figure}[htp]
\centering
\includegraphics[width=\textwidth]{rnn_2}
\caption{RNN的多种模式}
\label{fig:rnn_2}
\end{figure}
RNN相对于传统的神经网络,它允许我们对向量序列进行操作:输入序列、输出序列、或大部分的输入输出序列。如图\ref{fig:rnn_2}所示,每一个矩形是一个向量,箭头则表示函数(比如矩阵相乘)。输入向量用红色标出,输出向量用蓝色标出,绿色的矩形是RNN的状态。RNN根据输入输出对应关系可分为如下几类:
\begin{enumerate}
    \item 没有使用RNN的Vanilla模型,从固定大小的输入得到固定大小输出(比如图像分类)
    \item 序列输出(比如图片字幕,输入一张图片输出一段文字序列)
    \item 序列输入(比如情感分析,输入一段文字然后将它分类成积极或者消极情感)
    \item 序列输入和序列输出(比如机器翻译:一个RNN读取一条英文语句然后将它以法语形式输出)
    \item 同步序列输入输出(比如视频分类,对视频中每一帧打标签)。
\end{enumerate}

\begin{figure}[htp]
\centering
\includegraphics[width=\textwidth]{rnn_3}
\caption{使用单层全连接神经网络作为循环单元的RNN结构}
\label{fig:rnn_3}
\end{figure}

图\ref{fig:rnn_3}展示了一个使用最简单的循环体结构的循环神经网络,在这个循环体中只使用了一个类似全连接层的神经网络结构。下面将通过该图中所展示的神经网络来介绍循环神经网络前向传播的完整流程。

循环神经网络中的状态是通过一个向量来表示的,这个向量的维度也称为神经网络隐藏层的大小,设其为$h$。循环体中的神经网络的输入有两部分,一部分为上一时刻的状态$X_{t-1}$,另一部分为当前时刻的输入样本。对于时间序列数据来说,每一时刻的输入样例可以是当前时刻的数据;对于语言模型来说,输入样例可以是当前单词对应的词向量。
假设输入向量的维度为$x$,那么上图中循环体的全连接层神经网络的输入大小为$h+x$。也就是将上一时刻的状态与当前时刻的输入拼接成一个大的向量作为循环体中神经网络的输入。因为该神经网络的输出为当前时刻的状态,于是输出层的节点个数也为$h$(节点个数就是向量的维度,或者是隐藏层的大小),循环体中的参数个数为$(h+x)∗h+h(h+x)*h+h(h+x)∗h+h$个(这里可以理解为输入层有$h+x$个神经元,输出层有$h$个神经元,从而形成一个全连接的前馈神经网络,有$(h+x)*h$个权值,有$h$个偏置)。

同时循环体的神经网络输出不但提供给下一个时刻作为状态,同时还提供给当前的时刻作为输出。为了将当前时刻的状态转化为最终的输出,循环体还需要另外一个全连接神经网络来完成这个过程。这和卷积神经网络中最后的全连接层的意义是一样的。类似的,不同时刻用于输出的全连接神经网络中的参数也是共享的(参数一致)。

循环神经网络可以更好地利用传统神经网络结构所不能建模的信息,但同时,这也带来了更大的技术挑战——长期依赖(long-term dependencies)问题。因此,当前预测位置和相关信息之间的文本间隔就有可能变得很大。当这个间隔不断增大时,类似图\ref{fig:rnn_3}中给出的简单循环神经网络有可能丧失学习到距离如此远的信息的能力。或者在复杂语言场景中,有用信息的间隔有大有小、长短不一,循环神经网络的性能也会受到限制。

\subsection{长短期记忆与门控神经网络}

长短期记忆网络(long short term memory, LSTM)\cite{LSTM}的设计正是为了解决上述RNN的依赖问题,即为了解决RNN有时依赖的间隔短,有时依赖的间隔长的问题。其中循环神经网络被成功应用的关键就是LSTM。

\begin{figure}[htp]
\centering
\includegraphics[width=\textwidth]{rnn_4}
\caption{长短期记忆网络LSTM}
\label{fig:rnn_4}
\end{figure}

LSTM的第一步是决定要从上一个时刻的状态中丢弃什么信息,其本质是由一个sigmoid全连接的前馈神经网络的输出管理,将这种操作称为遗忘门(forget get layer)。这个全连接的前馈神经网络的输入由向量$h_{t-1}$​和$x_t$组成,输出是$f_t$,1表示能够通过,0表示不能通过。
\begin{equation}
    f_t = \sigma(W_f \cdot [h_{t-1},x_t]+b_f)
\end{equation}

第二步决定哪些输入信息要保存到神经元的状态中。首先是一个sigmoid层的全连接前馈神经网络,称为输入门(input gate layer),其决定了哪些值将被更新;然后是一个tanh层的全连接前馈神经网络,其输出是一个向量$\tilde{C_t}$,该向量可以被添加到当前时刻的神经元状态中;最后根据两个神经网络的结果创建一个新的神经元状态。
\begin{align}
    i_t &= \sigma(W_i \cdot [h_{t-1,x_t}]+b_i) \\
    \tilde{C_t} &= \tanh(W_C \cdot [h_{t-1},x_t]+b_C)
\end{align}

第三步就可以上一时刻的状态$C_{t−1}$更新为当前状态$C_t$了。上述的第一步的遗忘门计算了一个控制向量,此时通过这个向量过滤一部分$C_{t-1}$状态;上述第二步的输入门根据输入向量计算了新状态,此时可以通过这个新状态$\tilde{C_{t-1}}$和$C_{t−1}$状态更新$C_t$
\begin{equation}
    C_t = f_t*C_{t-1} + i_t*\tilde{C_t}
\end{equation}
​	
最后一步就是决定神经元的隐状态$h_t$	,此时的输出根据上述第三步的状态$C_t$进行计算:首先通过sigmoid层生成一个过滤向量;然后通过一个$\tanh$函数计算当前时刻的$C_t$;最后通过sigmoid层输出当前时刻的输出。
\begin{align}
    o_t &= \sigma(W_o [h_{t-1},x_t]+b_o) \\
    h_t &= o_t * \tanh(C_t)
\end{align}

GRU(Gated Recurrent Unit)\cite{GRU}是长短时记忆网络单元的一个变种,其主要优点继承了LSTM能够解决长期依赖的特点,同时结构更简单,训练与推断的计算复杂度更低,目前已在自然语言处理和其他序列分析领域得到了广泛应用。
\begin{figure}[htp]
\centering
\includegraphics[width=\textwidth]{gru}
\caption{门控神经网络单元(GRU)}
\label{fig:gru}
\end{figure}





\section{本章小结}
%分层存储和自然语言处理合并为一章:相关工作介绍,质量可以适当降低。
% 引言:概述分层存储特点-》分层存储管理的重要性->j
% 2.1 分层存储
% 	2.1.1 分层存储模型
% 	2.1.2 数据分类
% 	2.1.3 分层存储管理
% 		数据关联挖掘;N-GRAM模型;缺陷
% 2.2 自然语言处理相关技术
% 	2.2.1 词嵌入:突出词嵌入能表达语义关联的特性;可视化展现;
% 	2.2.2 循环神经网络:突出RNN在序列分析中的作用。举例LSTM用于内存访问预测的论文;

\chapter{理论模型}
\section{利用词嵌入技术构造新型元数据}
\subsection{文件名、目录名的向量化}
在Unix文件系统中,任意文件或目录在被创建时,均会被分配一个唯一的ID,也就是Inode序号,以此作为唯一的标识以便于后续各种文件操作。Inode序号只与文件创建先后相关,是一种one-hot的表示方式。那么能否效仿自然语言处理中词嵌入的思想,建立一种能够包含文件或目录之间关联性的表征方式?

文件系统层次结构规范(Filesystem Hierarchy Standard,FHS)\cite{fhs}是由Linux基金会在1994年发起,旨在规范Linux各发行版和其他类Unix系统下文件目录结构的业界统一标准,至今已发展演变到FHS-3.0(2015年)。在FHS定义的目录结构规范下,Linux操作系统的目录组织结构和命名受到了明确严格的约束,例如:

{\color{red}这里做成表格!}
\begin{itemize}
    \item /:根目录。
    \item /bin:系统执行文件目录。
    \item /boot:启动文件目录。
    \item /dev:驱动设备目录。
    \item /etc:系统配置文件目录。
    \item /lib, /usr/lib, /usr/local/lib:系统使用的函数库目录。
    \item ......
\end{itemize}

如果将一个文件的完整路径(如/usr/lib/python2.7/)视为一个句子,各级目录视为单词,我们假定类似于自然语言中的分布式假设同样成立,即:同一目录下的文件或目录具有类似的含义。直观上看,这个假设是合理的,例如/bin目录下的bash,rm,cp文件等均为可执行命令,/usr/lib/python2.7/目录下均为Python的库文件。在此假设成立的前提下,本节将介绍如何使用Skip-gram模型,对给定的Unix文件系统目录下所有文件名、目录名进行向量化。

\subsubsection*{语料库的生成}
众所周知,任何机器学习模型都离不开数据,自然语言处理领域的数据集通常被称为语料库(Corpus)。任何语言学的研究或者自然语言模型的建立都必须建立在大量的语料之上,否则无论是基于规则方法还是统计方法建立的模型都将失效。

为了建立文件路径相关的“语料库”,我们将从根目录(或文件系统的挂载点)开始,通过常规的遍历算法将此目录下所有文件、目录的完整路径逐行写入到一个文本文件,作为模型的训练数据集。语料库生成后,相应地可以得到一个词汇表$V$。
{\color{red}此处插入遍历算法}

\subsubsection*{利用Skip-gram模型训练文件和目录名的词向量}
给定总单词数量为$T$的语料库,其词汇表$\mathcal{V}$=\{\textit{bin,boot,dev,...}\},词汇数量为$W$。
%其中任意一个词可用其在词汇表中的序号表示:$w \in \{1,\dots,W\}$。
那么该语料库可表示为一个由$T$个词向量组成的序列:$\mathbf{w}_1, \mathbf{w}_2, \dots, \mathbf{w}_T$。Skip-gram模型的目标就是建立一个从词汇表到$d$维向量空间的映射$model:\mathcal{V} \rightarrow \mathbb{R}^d$,使以下对数极大似然函数达到最大:
\begin{equation}
    \label{eq:origin_object}
    \frac{1}{T}\sum_{t=1}^T \sum_{c \in \mathcal{C}_t} \log p(\mathbf{w}_c | \mathbf{w}_t),
\end{equation}
取负号得到损失函数:
\begin{equation}
    L(\theta)= -\frac{1}{T}\sum_{t=1}^T \sum_{c \in \mathcal{C}_t} \log p(\mathbf{w}_c | \mathbf{w}_t),
\end{equation}

其中,$\mathcal{C_t}$表示某中心词$\mathcal{w}_t$的上下文中出现过的词的集合。以路径\textit{/usr/local/lib/python}为例,若取上下文窗口大小为1,那么中心词$\mathcal{w}_t$=\textit{lib}的上下文集合$\mathcal{C_t}$=\{\textit{local,python}\},在文件系统的情境下,此处“上下文”隐含的语义是指\textit{usr},\textit{python}分别与\textit{lib}有着父子目录的关系。对于任意样本$(\mathbf{\mathcal{w}}_t,\mathbf{\mathcal{w}}_c)$,我们可用归一化函数softmax来定义单词$\mathcal{w}_c$在中心词$\mathbf{\mathcal{w}}_t$的上下文中出现的条件概率:
\begin{equation}
    \label{eq:softmax}
    p(\mathbf{w}_c | \mathbf{w}_t)=\frac{ e^{ \mathbf{w}_t^{\top} \mathbf{w}_c} }{ \sum_{j=1}^W e^{\mathbf{w}_t^{\top} \mathbf{w}_j}} 
\end{equation}

\begin{figure}[htp]
\centering
\includegraphics[width=\textwidth]{word2vec_nn}
\caption{Skip-gram模型网络}
\label{fig:word2vec_nn}
\end{figure}
如图\ref{fig:word2vec_nn},Skip-gram模型是一个三层神经网络,输入为词汇表中单词的初始向量,即$W$维的one-hot编码。隐层由$d$个神经元组成,没有激活函数,只有权值。训练收敛后隐层的$W\times d$的权值矩阵就是词汇表内所有词向量的集合。输出层为上文所述的softmax函数,最终结果是一个$W$维的向量,每个分量表示对应的单词与输入词存在上下文关系的概率。

当语料库规模较大,词汇表内单词较多时,采用softmax函数的计算复杂度过高。一种优化方式是使用层次化softmax(Hierarchical softmax)\cite{Hierarchical_softmax}。另一种计算复杂度更低,且同样能保证对原有softmax层拟合精度的优化方式是负采样(Negative sampling)。该方法将原来的预测上下文的问题转化为一系列独立的二分类问题,即,在选定中心词后,对词汇表中其他单词依次判定是否在中心词附近出现。


对任意中心词$\mathbf{w}_t$,用交叉熵损失函数(Cross-entropy Loss)代替原来的损失函数
\begin{equation}
    \label{eq:loss_of_t}
    L_t(\theta) = -\left( 
        \sum_{c \in \mathcal{C}_t} \log(p(\mathbf{w}_c | \mathbf{w}_t))+\sum_{c \in \mathcal{N}_t} \log(1-p(\mathbf{w}_c | \mathbf{w}_t)) 
    \right)
\end{equation}
其中$\mathcal{C}_t$表示中心词$w_t$的上下文单词集合(正样本),$\mathcal{N}_t$表示词汇表中,与$w_t$不存在上下文关系的单词(负样本)中随机抽取的若干噪声词。

由于任意单词与中心词是否上下文被视为独立事件,可用sigmoid函数拟合条件概率
\begin{equation}
    p(\mathbf{w}_c | \mathbf{w}_t) = \frac{1}{1+e^{-\mathbf{w}_t^{\top} \mathbf{w}_c}} = \sigma(\mathbf{w}_t^{\top} \mathbf{w}_c)
\end{equation}

代入\ref{eq:loss_of_t}并按$t$累加求平均,得到最终的损失函数:
\begin{equation}
    L(\theta) = \frac{1}{T}\sum_{t=1}^{T} \left[
        \sum_{c \in \mathcal{C}_t} \log(\sigma(\mathbf{w}_t^{\top} \mathbf{w}_c)) + \sum_{c \in \mathcal{N}_t} \log(\sigma(-\mathbf{w}_t^{\top} \mathbf{w}_c))
    \right]
\end{equation}


%\begin{equation}
%    \label{eq:neg}
%    \log(1+e^{-\mathbf{w}_t^{\top} \mathbf{w}_c})+ \sum_{n \in \mathcal{N}_{t,c} }\log(1+ e^{\mathbf{w}_t^{\top} \mathbf{w}_n})
%\end{equation}
%
%
%用sigmoid函数$\sigma(x) = \log(1+e^{-x})$与公式\ref{eq:neg}代入目标函数\ref{eq:origin_object}得到最终的目标函数:
%\begin{equation}
%    \frac{1}{T} \sum_{t=1}^{T} \left[ \sum_{c \in \mathcal{C}_t} \sigma(\mathbf{w}_t^{\top} \mathbf{w}_c) + \sum_{n \in \mathcal{N}_{t,c}} \sigma(-\mathbf{w}_t^{\top} \mathbf{w}_n) \right]
%\end{equation}

经过以上优化后,Skip-gram模型训练的计算复杂度大大缩小。隐层的$W\times d$的权值矩阵$\theta$通过常规的随机梯度下降训练收敛后,作为最终词汇表$\mathcal{V}$的词向量模型。

\subsection{引入子词模型}
%阐述文件系统内命名与人类自然语言的区别,与前文FastText中的子词模型呼应
\subsection{路径向量}
%注意区分绝对路径、工作路径、挂载点等造成的差异
\section{基于门控神经网络的热点数据识别}
\subsection{问题描述}
%设文件系统进行文件向量化处理后,所有文件向量($d$维)组成的集合记为$\mathcal{F} \subset \mathbb{R}^d$。在某工作负载的生命周期内对其文件访问进行追踪,将追踪日志转化为$d$维路径向量序列$\mathcal{S}=\{\mathbf{x}_1, \mathbf{x}_2,\dots, \mathbf{x}_T\}$。
%设上下文窗口大小为$N$,即
%在任意时刻$t$,追踪模块对前$N$次访问构成的日志序列$\{\mathbf{x}_{t-N+1}, \dots, \mathbf{x}_t\}$记为$\mathcal{L_t}$。
%定义该时刻的上下文序列$\mathcal{C}_t = \{ \mathbf{x}_{t-N+1}, \dots, \mathbf{x}_{t+N} \}$为正类样本(热文件),其余所有文件$\mathcal{N}_t = \mathcal{F} - \mathcal{C}_t$为负样本(冷文件)。
%
%在以上定义下,热点数据识别可转化为如下二分类问题的求解,其目标是:建立模型$M$,以$\mathcal{L}_t$为输入,计算这段访问序列所隐含的访问模式$\mathbf{h}_t$($d$维向量)。同时定义函数$f:\mathbb{R}^d \times \mathcal{F} \rightarrow [0,1]$,其含义为:访问模式$\mathbf{h}_t$下,文件$\mathbf{x}$是热点文件的概率。最后,为保证冷热文件的分类不会因为计算结果频繁扰动,设置合理的阈值$0<\alpha<\beta<1$来判定冷热。

设文件系统进行文件向量化处理后,所有文件向量($d$维)组成的集合记为$\mathcal{F} \subset \mathbb{R}^d$。在某工作负载的生命周期内对其文件访问进行追踪,将追踪日志转化为$d$维路径向量序列$\mathcal{S}=\{\mathbf{x}_1, \mathbf{x}_2,\dots, \mathbf{x}_T\}$。

在给定上述数据集的条件下,热点数据识别可转化为如下动态二分类问题求解:建立模型计算任意时刻$t$所隐含的访问模式$\mathbf{h}_t$。同时定义函数$f:\mathbb{R}^d \times \mathcal{F} \rightarrow [0,1]$,其含义为:访问模式$\mathbf{h}_t$下,文件$\mathbf{x}$是热点文件的概率。最后,为保证冷热文件的分类不会因为计算结果频繁扰动,设置合理的阈值$0<\alpha<\beta<1$来判定冷热。


\subsection{基于门控神经网络的模型设计}

在当前大数据背景下,海量小文件访问的读写性能是制约存储系统整体性能的瓶颈之一。在针对此类工作负载进行访问模式分析与热点数据识别时,上节所定义的问题规模将会十分巨大:应用生命周期内累积的访问序列过长,模型应具备长期记忆的功能。为此,本文采用门控神经网络单元作为访问模式的分析模型。

{\color{red}如图},

\subsubsection*{1.输入层设计}

设上下文窗口大小为$N$。在任意时刻$t$,我们将最近$N$次文件访问构成的日志序列$\{\mathbf{x}_{t-N+1}, \dots, \mathbf{x}_t\}$作为输入序列$\mathcal{L_t}$。

\subsubsection*{2.隐层设计}
如图所示,本文采用单层GRU来计算某一时刻隐含的文件访问模式$h_t$。前向传播的具体计算过程如下:

在时刻$t$, 我们用$h_t^{(j)}$来表示隐状态$\mathbf{h}_t$的第$j$个分量,由前一时刻隐状态$h_{t-1}^{(j)}$与候选隐状态$\tilde{h}_t^{(j)}$加权求和计算:
\begin{equation}
    h_t^{(j)} = (1-z_t^{(j)}) h_{t-1}^{(j)} + z_t^{(j)}\tilde{h}_t^{(j)},
\end{equation}
其中$z_t$为更新门(update gate),由当前时刻输入$\mathbf{x_t}$和上一刻的隐状态$\mathbf{h_{t-1}}$计算获得:
\begin{equation}
z_t^{(j)} = \sigma(W_z \mathbf{x}_t + U_z \mathbf{h}_{t-1})^{(j)}
\end{equation}
候选隐状态$\tilde{h}_t^{(j)}$的计算如下:
\begin{equation}
    \tilde{h}_t^{(j)} = \tanh(W \mathbf{x}_t + U (\mathbf{r}_t \odot \mathbf{h}_{t-1}))^{(j)}
\end{equation}
此处$\mathbf{r}_t$是重置门(reset gate),运算符号$\odot$表示行列相同的矩阵之间的逐元素相乘(Hadamard product)。此处表示$\mathbf{r}_t$的第$j$个元素与隐状态$h_{t-1}$的第$j$个元素相乘。

重置门$r_t$由以下表达式求得:
\begin{equation}
    r_t^{(j)} = \sigma(W_r \mathbf{x}_t + \mathbf{U}_r \mathbf{h}_{t-1})^{(j)}
\end{equation}

\subsubsection*{3.输出层设计}

如上文所述,我们构造了单层GRU来计算隐状态$h_t$,为了赋予$h_t$识别热点文件的功能,我们构造函数$f:\mathbb{R}^d \times \mathcal{F} \rightarrow [0,1]$用于计算在状态$\mathbf{h}_t$下,文件$\mathbf{x}_t$是热数据的条件概率:
\begin{align}
    \begin{split}
    p(\mathbf{x} | \mathbf{h}_t) &= f(\mathbf{h}_t,\mathbf{x}) \\
                                &= \sigma(\mathbf{h}_t^{\top} \mathbf{x}) \\
                                &= \frac{1}{1+e^{-\mathbf{h}_t^{\top} \mathbf{x}}}
    \end{split}
\end{align}

与常规的文本分类任务不同,文件访问序列组成的样本不带标签,或者说冷、热标签是随着时刻$t$动态改变的。在实际应用场景下,$t$时刻前访问尚未结束的文件,以及接下来即将访问的文件都是热数据,然而缓存容量是有限的,可以被纳入热数据的总量应与缓存容量呈正相关。为简化模型起见,此处设定与时间无关,只与缓存容量相关的超参数$M$,将文件序列$\{ x_{t-M+1},\dots,x_t,\dots,x_{t+M} \}$共计$2*M$条文件标记为正类样本集$C_t$,其余文件为冷数据。

通常冷文件数量近似于文件总数$|\mathcal{F}|$,考虑到实际文件系统中总文件数量巨大,将所有冷文件纳入负类样本将带来巨大的计算量。为此,本文借鉴Skip-gram算法中采用的负采样(Negtive sampling)的思想,对冷文件集随机采集$K$个样本组成负类样本集$\mathcal{N}_t$。

为训练该模型,采用二分类最常用的交叉熵损失函数:
\begin{equation}
    L(\theta) = \frac{1}{T-2M}\sum_{t=M}^{T-M+1} \left[
        \sum_{x \in \mathcal{C}_t} \log(p(x_i | h_t)) + 
        \sum_{x \in \mathcal{N}_t} \log(-p(x_i | h_t))
    \right]
\end{equation}

其中,待训练的参数$\theta$为GRU中各个部件的权重矩阵$W_z, U_z, W, U, W_r, U_r$,训练采用时序后向传播算法(Backpropagation Through Time)。要设置和调优的超参数包括输入序列长度$N$,与缓存容量正相关的热文件样本数$M$,以及负样本采样数$K$。

%指标以命中率、错判率为主,即准确率召回率?
\section{本章小结}

% 基于自然语言处理的文件分类模型
% 引言:放个总框图,直观展示两项主要工作
% 3.1	基于词嵌入模型的文件关联分析
% 	3.1.1 具体模型(skip-gram+subword)
% 		词与词之间关联;
% 		线性相加后表达文件之间关联。
% 		注意加图:skipgram的图
% 	3.1.2 词向量模型可视化:
% 		tnse
% 3.2	基于RNN的冷热分类模型
% 	3.2.1 基于单层GRNN的冷热分类模型
% 	3.2.2 分类阈值讨论


\chapter{主动预取框架设计}
\section{Trace模块}
\section{访问模式识别模型建立}
\section{运行时缓存管理}
\section{本章小结}

% 引言:总框图,描述
% 4.1 客户端设计
% 4.2 服务端设计

\chapter{实验设计与结果分析}
Results
\section{实验环境与工作负载}

\section{实验数据采集与预处理}
\section{实验结果与分析}
\subsection{文件与目录名向量化}
\subsection{门控循环神经网络模型训练与仿真测试}
\section{本章小结}
%gluster tiering 模块实验
%针对工作负载的文件向量化实验(可视化)
%采用循环神经网络进行文件冷热分类
\chapter{总结与展望}
在当前大数据与高性能计算蓬勃发展的背景下,分层存储系统在计算机领域的应用越来越广泛。尤其是伴随着NVMe、SSD等高性能存储介质相关技术的飞速发展,分层存储管理中的热数据识别和数据迁移技术成为分层存储管理优化的关键。国内外对于相关领域的优秀成果不断涌现,尤其是主动数据迁移预取方面,许多学者对文件访问模式的本质进行了重要的探索,如数据挖掘领域的频繁序列挖掘算法,自然语言处理领域的N-gram模型和循环神经网络等模型均在该领域有了立足之地。

目前分层存储管理优化方式主要包括两方面:一是挖掘文件的关联性(Block Correlation),二是动态追踪应用的I/O行为并加以分析,实现I/O行为预测(热数据识别),以指导数据迁移模块进行主动的数据预取和缓存替换。以往的研究具有局限性,例如,对文件的关联性挖掘局限于目录树结构表现出的文件关系,没有对文件或目录命名隐含的语义关系予以分析;I/O行为的分析通常只针对短期内的访问规律,没有在长时间跨度上进行分析。为解决上述问题,探索新的分层管理优化方式,本文做了以下几方面的工作:

1)文件的向量化表示。数据块关联(Block Correlation)在文件系统中广泛存在,且这种数据间的联 系通常比较稳定,在目录结构不发生变化的情况下,一般不受工作负载的运行状 态影响。反过来,工作负载访问文件的部分规律是由文件之间的固有关系决定的。 如果能显示地挖掘出文件之间的联系,可以为存储系统的数据分布策略、迁移策 略等提供帮助。本文从文件之间语义关系入手,建立了词嵌入模型将文件和目录映射为向量,将文件之间的关联分析转化为向量运算。

2)基于循环神经网络的文件分类模型。本文对冷热文件分类任务进行了形式化定义,在文件向量化表示的基础上,建立了GRU循环神经网络对应用的I/O序列进行分析和文件分类。

3)在上述两项工作的基础上,本文在GlusterFS架构的基础上设计了一套分层存储管理框架,该框架对Tiering模块、CTR模块以及GFDB数据库等进行了改进设计,将文件向量作为额外的元数据,并将循环神经网络嵌入到CTR模块中代替传统的LRU缓存算法指导数据迁移。

4)以开源工程编译任务作为工作负载,设计相关实验对上述两项模型进行验证。


基于上述模型设计,本课题的实验部分选用开源工程编译任务作为工作负载,设计了一套从实验数据采集到模型测试的具体实施方案,给出了相应的评估标准,最后就实验结果进行了充分的分析:基于循环神经网络的文件分类模型在面向单一进程编译任务时取得了一定的效果,但面对更复杂的多任务编译负载时性能较差。

本课题涉及的交叉知识较多,在课题研究的过程中深感自己的知识储备远远不够,工程能力也尚有欠缺。本课题还有许多不足和需要做的工作:

1)进一步完善课题内关于自然语言处理部分的理论框架。

2)要深入分析目前模型存在的缺陷和不合理之处,对模型设计作进一步完善,

3)将研究的范围作进一步扩展,如对象文件系统、并行文件系统等。

4)完成GLusterFS上分层管理模块的工程实现。



%%% Local Variables:
%%% mode: latex
%%% TeX-master: "../main"
%%% End:

\begin{ack}
时光荏苒,我的硕士生涯已接进尾声。这几年的时光既漫长又短暂,其中充满了酸甜苦辣,更有收获和成长。几年来,感谢陪我一起度过美好时光的每位尊敬的老师和亲爱的同学,正是你们的帮助,我才能克服困难,正是你们的指导,我才能解决疑惑,直到学业的顺利完成。

本人的学位论文是在我的两位恩师周恩强教授和刘杰教授的殷切关怀和耐心指导下进行并完成的,衷心感谢他们对我的淳淳教诲和悉心关怀。从课题的选择、项目的实施,直至论文的最终完成,恩师始终给予我耐心的指导和支持,我取得的每一点成绩都凝聚着恩师的汗水和心血。恩师开阔的视野、严谨的治学态度、精益求精的工作作风,深深地感染和激励着我,在此谨向他们致以衷心的感谢和崇高的敬意。
  
感谢实验室的师弟师妹们与我一道分享他们青春的快乐!在此还要对实验室所有师兄弟姐妹们在平时开展相关工作中的支持和帮助一并表示感谢。无论在炎热的夏天,还是寒冷的冬季,他们不辞劳苦地为我提供无私的帮助,没有他们的帮助就没有这篇论文的顺利完成。
  
感谢国防科技大学给我提供的平台和机会。今后我会更加努力钻研学术,也更加清醒地意识到必须做终生学习型的人。感谢我的家人常年对我的支持和理解!他们是最爱我的人,也是我亏欠最多的人,他们默默的奉献是我多年来的支持和动力。
  
最后,我要向百忙之中参与审阅、评议本论文各位老师、向参与本人论文答辩的各位老师表示由衷的感谢!人生的每个阶段都值得好好珍惜,这段美好岁月,因为有你们的关心和帮助,我很幸福。我会更加勤奋学习、认真研究,我会努力做得更好,我想这也是我能给你们的最好的回报吧。把最美好的祝福献给你们,愿永远健康、快乐!
\end{ack}


\cleardoublepage
\phantomsection
\addcontentsline{toc}{chapter}{参考文献}
\bibliographystyle{bstutf8}
\bibliography{ref/thesis}

\begin{resume}
%该论文作者在学期间取得的阶段性成果(学术论文等)已满足我校硕士学位评阅相关要求。为避免阶段性成果信息对专家评价学位论文本身造成干扰,特将论文作者的阶段性成果信息隐去。
  \section*{发表的学术论文} % 发表的和录用的合在一起

  \begin{enumerate}[label={[\arabic*]},itemsep=0pt,parsep=0pt,labelindent=26pt,labelwidth=*,leftmargin=0pt,itemindent=*,align=left]
   %[label=\textbf{[\arabic*]},itemindent=*, align=left] %老版本缩进对齐
   
  %\addtolength{\itemsep}{-.36\baselineskip}%缩小条目之间的间距,下面类似
  \item Hui Chen, Enqiang Zhou, Jie Liu, Zhicheng Zhang, et al. An RNN Based Mechanism for File Prefetching. 18th Distributed Computing and Applications for Business Engineering and Science.

  \end{enumerate}
\end{resume}

% 最后,需要的话还要生成附录,全文随之结束。
%\appendix
\backmatter
%\input{data/appendix01}

\end{document}

